
\begin{frame}[fragile]{Yocto Image}
	\begin{multicols}{2}
		\begin{lstlisting}[title=mume-dev-image.bb,frame=single,language=bitbake]
LICENSE = "MIT"

inherit core-image
inherit populate_sdk_qt5

IMAGE_INSTALL = " \
  packagegroup-core-boot \
  @\ldots@-core-ssh-openssh \
  packagegroup-mume-common \
  packagegroup-dev-mume \
  \
  mupro-start \
"

IMAGE_FEATURES += " \
  package-management \
  debug-tweaks \
"
		\end{lstlisting}
		\begin{lstlisting}[title=packagegroup-dev-mume.bb, frame=single, numbers=right, language=bitbake]
SUMMARY = "developer tools for MUME"
LICENSE = "MIT"

inherit packagegroup

RDEPENDS_${PN} = "\
  bash \
  devmem2 \
  e2fsprogs \
  htop \
  nginx \
  openssh-sftp \
  perf \
  qtbase \
  strace \
  time \
  @\ldots@
		\end{lstlisting}
	\end{multicols}
\end{frame}

\begin{frame}[fragile]{Yocto Image bauen}
	\begin{lstlisting}[frame=single,language=bash]
$ bitbake mume-dev-image
	\end{lstlisting}
	\begin{table}
		\caption{tmp/deploy/images/beaglebone/}
		\begin{tabular}{ll}
			\hline MLO & stage 1 loader \\ 
			\hline u-boot.img & stage 2 loader \\ 
			\hline uEnv.txt & u-boot Konfiguration \\ 
			\hline zImage & Kernel \\ 
			\hline zImage-bonegreen-mume.dtb & Device Tree \\ 
			\hline mume-dev-image-beaglebone.tar.bz2 & rootfs \\ 
			\hline 
		\end{tabular} 
	\end{table}
\end{frame}

\begin{frame}{rootfs}
	beinhaltet
	\begin{itemize}
		\item userspace
		\item ev. Kernel \& Device Tree
		\item nicht bootloader
	\end{itemize}
\end{frame}

\begin{frame}[fragile]{Device Tree}
	\begin{multicols}{2}
		\begin{itemize}
			\item Linux kennt Hardware nicht
			\item Device Tree beschreibt Hardware
			\item Linux lädt Treiber anhand Device Tree
		\end{itemize}
		\pagebreak
		\begin{lstlisting}[frame=,numbers=none,escapeinside={|}{|},language=]
/ {
  compatible = "ti,am33xx";

  spi0: spi@48030000 {
    compatible = "ti,omap4-mcspi"|\footnote{Mapping zu Treiber}|;
    status = "disabled"|\footnote{aktivieren durch status="okay"}|;
    reg = <0x48030000 0x400>|\footnote{Register-Position, siehe Datenblatt des SOC}|;
    interrupts = <65>;
    dmas = <&edma|\footnote{Referenz zum DMA Device-Tree Knoten}| 16 &edma 17 >;
    |\dots|
  };
|\dots|
		\end{lstlisting}
	\end{multicols}
\end{frame}

\begin{frame}{Boot Process \cite{OMAPBootloaderOverview}}
	\begin{tabular}{|l|l|l|}
		\hline typ & name & funktion \\ 
		\hline system startup & ROM Code & minimal hardware initialisierung \\ 
		& & in boot devices nach image suchen \\
		& & stage 1 loader ins ram laden und ausführen \\
		
		\hline stage 1 loader & x-loader (u-boot) & pin muxing \\ 
		& & clock und memory initialisieren \\
		& & stage 2 loader ins ram laden und ausführen \\
		
		\hline stage 2 loader & u-boot & platform initialisierung (USB, Netzwerk, \ldots) \\
		& & boot menu / Kommandozeile anzeigen \\
		& & Kernel und Device-Tree ins ram laden und ausführen \\
		
		\hline kernel & Linux & Treiber für Hardware laden \\ 
		& & root file system mounten \\
		& & init process starten \\
		
		\hline init & Systemd & Abhängigkeiten zwischen Services auflösen \\ 
		& & Services starten \\
		& & Services überwachen \\
		
		\hline 
	\end{tabular} 
\end{frame}

\begin{frame}[fragile]{SDK bauen}
	\begin{lstlisting}[frame=single,language=bash, breaklines=true]
$ bitbake mume-dev-image -c populate_sdk
$ ls -sh tmp/deploy/sdk/
695M mume-glibc-x86_64-mume-dev-image-cortexa8hf-vfp-neon-toolchain-2.0.sh
	\end{lstlisting}
	\begin{itemize}
		\item Cross-Compiler
		\item Entwicklerpakete (header, \ldots)
		\item root file system (libraries, \ldots)
		\item Paketmanager
	\end{itemize}
\end{frame}
